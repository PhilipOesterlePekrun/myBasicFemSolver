\documentclass[12pt,a4paper]{article}
%\usepackage[a4paper, margin=1.5cm]{geometry} %//##

\newcommand{\thesistype}{Bachelor's Thesis}
\newcommand{\thesistitle}{Algebraic Multigrid Preconditioning for Contact and Mesh Tying Problems in Saddle Point Formulation}
\newcommand{\shorttitle}{AMG Preconditioning for Contact and Mesh Tying Problems in Saddle Point Formulation}
\newcommand{\Author}{Philip Oesterle-Pekrun}
\newcommand{\Supervisor}{SupervisorName}
\newcommand{\DateOfSubmission}{14.04.2025}

% packages included by LNM template
\usepackage{fancyhdr}
\usepackage{mdwlist}
\usepackage{dsfont}
\usepackage{graphicx}
\usepackage{amssymb}
\usepackage{amsmath}
\usepackage{amsthm}
\usepackage{ifthen}
% commented out in template:
%\usepackage{epstopdf}
%\epstopdfsetup{update} % only regenerate pdf files when eps file is newer

% MY PACKAGES (some may be redundant):{
	\usepackage{graphicx} % Required for inserting images
	\usepackage{lipsum} % For dummy text
	
	\usepackage{hyperref} % For clickable links
	\usepackage{xcolor} % For colored links
	
	\usepackage{tocloft} % For controlling ToC appearance
	\usepackage[toc,page]{appendix}
	
	\usepackage{tabto} % For dotted lines automatically//#?
	\usepackage{titlesec} % Needed for ``1 Introduction'' rather than ``Chapter 1 \\
	%Introduction''//#?
	%\usepackage{mathptmx}  % Times New Roman equivalent BUT DOES NOT WORK WITH BM
	\usepackage{newtxtext} % Working Times New Romn equivalent
	\usepackage{newtxmath} %
	%\usepackage{unicode-math} % Does not work
	\usepackage{bm} % Beter than \mathbf because also greek letters can be made bold
	
	\usepackage{chngcntr} % To number figures chapter.count
	\counterwithin{figure}{section}
	
	\usepackage{url} % For MueLu/Trilinos citation url
	
	\usepackage{subcaption} % For subfigures
	\usepackage{caption} %
	
	\usepackage{tikz} % For simple 3D figures and such
	\usetikzlibrary{patterns, arrows, shapes.geometric}
	\usetikzlibrary{3d, fadings}
	
	\usepackage{arydshln} % For dashed lines in block matrices
	
	%\usepackage{etex} % For more resources (stops random errors)
	
	%%\usepackage{cite} % For correct number ordering; INCOMPATIBLE WITH BIBLATEX BIBER WHATEVER
	
	\usepackage{algorithm} % For algorithms
	\usepackage{algpseudocode} %
	%\usepackage[ruled,vlined,linesnumbered]{algorithm2e} % For the vertical lines; WARNING: CANT BE LOADED TOGETHER WITH THE OTHER TWO
	
	\usepackage{caption} %
	
	\usepackage{accents} % For \underaccent
	
	\usepackage{float} % For fig and table and whatever placement
	
	\usepackage[absolute,overlay]{textpos} % For title image
	
	\usepackage[percent]{overpic} % Put images over stuff
	
	\usepackage{longtable} % Tables
	
	\usepackage[style=numeric,backend=biber]{biblatex} % For capitalized references titles
	\addbibresource{00_Bibliography.bib}
	
	
	%\usepackage{makecell} % For \Xhline{1pt} in tables
	%\usepackage{multirow, booktabs}
	%\usepackage{siunitx}
	%\usepackage{tabularray} % none of these work...
	
	
	\theoremstyle{remark} % Remarks from amsthm
	\newtheorem{remark}{Remark}[section] %
	%}

%%%%%%%%%%%%%%%%%%%%%%%%%%%%%%%%%%%%%%%
% MY NEWCOMMANDS:
\newcommand{\nablaR}{\nabla^{\mathrm R}} % right-hand nabla
\newcommand{\nablaRwrt}[1]{\nabla_{#1}^{\mathrm R}} % right-hand nabla wrt vector (1st order tensor)
\newcommand{\nablaL}{\nabla^{\mathrm L}} % left-hand nabla
\newcommand{\nablaLwrt}[1]{\nabla_{#1}^{\mathrm L}} % left-hand nabla wrt vector (1st order tensor)
\newcommand{\bu}[1]{\bm{\mathrm #1}} % bold upright symbol (for non-tensor matrices and vectors)




\makeatletter
\newcommand{\thickhline}{\noalign{\ifnum0=`}\fi\hrule height 1pt \futurelet\reserved@a\@xhline} % might lead to issues...
\makeatother






% suggested math abbreviations - extend at will
%\renewcommand{\vec}[1]{\boldsymbol{#1}}         % for vectors
%\newcommand{\mat}[1]{\boldsymbol{#1}}           % for matrices

%\newcommand{\dd}{\mathrm{d}}                    % differential d
%\newcommand{\pd}{\partial}                      % partial differentiation d

\newcommand{\tsum}{{\textstyle\sum\limits}}     % for small sums in large equations
\newcommand{\pfrac}[2]{\frac{\pd #1}{\pd #2}}   % \pfrac{f(x,y)}{x} for partial derivative of f(x,y) with respect to x
\renewcommand{\dfrac}[2]{\frac{\dd #1}{\dd #2}} % \dfrac{f(x,y)}{x} for total derivative of f(x,y) with respect to x
%\newcommand{\grad}{\nabla}                      % for the gradient
%\newcommand{\norm}[1]{\| #1 \|}                 % \norm{x} for the norm of x
%\newcommand{\abs}[1]{| #1 |}                    % \abs{x} for the absolute value of x

% suggested text abbreviations - extend at will
\newcommand{\comment}[1]{ }  % put aroung regions that are not to be compiled

%
%\newcommand{\insertblankpage}{\mbox{}\thispagestyle{empty}\addtocounter{page}{-1}\newpage} % original
\newcommand{\insertblankpage}{\mbox{}\thispagestyle{empty}\newpage} % mine
\newcommand{\BibTeX}{$\mathrm{B{\scriptstyle{IB}} \! T\!_{\displaystyle E} \! X}$}

% source of all figures
\graphicspath{{./}}
% This is the folder where your figures should be. 
% You can define more folders by inserting  {./additionalPath/}.

% Leere Fu''llseiten
\makeatletter
\renewcommand{\cleardoublepage}{\clearpage\if@twoside \ifodd\c@page\else
	\hbox{}
	\vspace*{\fill}
	\thispagestyle{empty}
	\newpage
	\if@twocolumn\hbox{}\newpage\fi\fi\fi}
\makeatother

%%%%%%%%%%%%%%%%%%%%%%%%%%%%%%%%%%%%%%%%%%%%%%%%%%%%%%%%%%%%%%%%%%%%%%%%%%%%%%%
%%% page layout %%%%%%%%%%%%%%%%%%%%%%%
%%%%%%%%%%%%%%%%%%%%%%%%%%%%%%%%%%%%%%%%%%%%%%%%%%%%%%%%%%%%%%%%%%%%%%%%%%%%%%%
% for details see, e.g. ``http://en.wikibooks.org/wiki/LaTeX/Page_Layout''
\setlength{\topmargin}{0mm}                % space above header (excluding \voffset)
\setlength{\headheight}{12pt}              % height of header
\setlength{\headsep}{8mm}                  % distance between header and text
\setlength{\textheight}{230mm}             % 
\setlength{\footskip}{10mm}                % distance between text and footer, including footer itself

\setlength{\evensidemargin}{0mm}           % distance to inner edge of left page (if twosided)
%\setlength{\oddsidemargin}{8mm}            % distance to inner edge of right page (or both if onesided)
\setlength{\oddsidemargin}{0mm}            % mine //##
\setlength{\textwidth}{160mm}              % 
\setlength{\marginparsep}{0mm}             % 

\setlength{\voffset}{0mm}                  % 
\setlength{\hoffset}{0mm}                  % 

% suggested headers and footers
%\pagestyle{fancyplain}
\fancyhf{}
\fancyfoot[C]{\fancyplain{}{\bfseries\thepage}}
\fancyhead[RO]{\fancyplain{}{\Author}}
\fancyhead[LO]{\fancyplain{}{\thesistype:\ \shorttitle}}

\newenvironment{changemargin}[2]{%
	\begin{list}{}{%
			\setlength{\topsep}{0pt}%
			\setlength{\leftmargin}{#1}%
			\setlength{\rightmargin}{#2}%
			\setlength{\listparindent}{\parindent}%
			\setlength{\itemindent}{\parindent}%
			\setlength{\parsep}{\parskip}%
		}%
		\item[]}{\end{list}}

\usepackage{setspace}
\usepackage{helvet}

% Keep the text font unchanged
%\usepackage{lmodern}

% For better math font styling
\usepackage{newtxmath}
%\renewcommand{\familydefault}{\sfdefault}


%%%%%%%%%%%%%%%%%%%%%%%%%%%%%%%%%%%%%%%%%%%%%%%%%%%%%%%%%%%%%%%%%%%%%%%%%%%%%%%
%%% MY CONFIGS BEFORE BEGIN DOCUMENT %%%%%%%%%%%%%%%%%%%%%%%
%%%%%%%%%%%%%%%%%%%%%%%%%%%%%%%%%%%%%%%%%%%%%%%%%%%%%%%%%%%%%%%%%%%%%%%%%%%%%%%
\numberwithin{equation}{section} % Equations numbered as (Section.Number)


%%%%%%%%%%%%%%%%%%%%%%%%%%%%%%%%%%%%%%%%%%%%%%%%%%%%%%%%%%%%%%%%%%%%%%%%%%%%%%%%
%%%%%%%%%%%%%%%%%%%%%%%%%%%%%%%%%%%%%%%%%%%%%%%%%%%%%%%%%%%%%%%%%%%%%%%%%%%%%%%%
%%%%  	BEGIN DOCUMENT
%%%%%%%%%%%%%%%%%%%%%%%%%%%%%%%%%%%%%%%%%%%%%%%%%%%%%%%%%%%%%%%%%%%%%%%%%%%%%%%%
%%%%%%%%%%%%%%%%%%%%%%%%%%%%%%%%%%%%%%%%%%%%%%%%%%%%%%%%%%%%%%%%%%%%%%%%%%%%%%%%
\begin{document}
	
	
%%%%%%%%%%%%%%%%%%%%%%%%%%%%%%%%%%%%%%%%%%%%%%%%%%%%%%%%%%%%%%%%%%%%%%%%%%%%%%%%
%%% title %%%%%%%%%%%%%%%%%%%%%%%
%%%%%%%%%%%%%%%%%%%%%%%%%%%%%%%%%%%%%%%%%%%%%%%%%%%%%%%%%%%%%%%%%%%%%%%%%%%%%%%%

%%%
% title page
\comment{
\begin{titlepage}
	\thispagestyle{empty}
	%///\vspace{-1.5cm}
	\vspace{-1.5cm}
	\includegraphics[height=1.5772cm]{fig/lnm}
	\hfill
	\includegraphics[height=1.5772cm]{fig/tum_text}
	%\vspace{-1.5cm}%////I added this
	\vfill
	
	\begin{center}
		\Huge{\thesistitle}
		\\
		\vspace{0.2cm}
		\LARGE{\Author}
		\\
		\vspace{0.2cm}
		\large{\thesistype}
	\end{center}
	% \vfill
	\vspace{0.1cm}
	%
	% und an diese Stelle kommt ein tolles Bild aus der eigenen Arbeit
	\begin{center}
		\includegraphics[width=10cm]{fig/titlePageImage2.png}
		%\includegraphics[width=10cm]{fig/titlePageSecondImage1.png}
	\end{center}
	%
	% \vspace{0.1cm}
	\vfill
	%
	\vspace{-0cm}%////I added this
	\begin{minipage}[c]{1.0\textwidth}
		\centering
		{\large \scshape Supervisor:}\\
		{Dr.-Ing. Matthias Mayr}\\
		\texttt{matthias.mayr@unibw.de}\\
		{M.Sc. Gil Robalo Rei}\\
		\texttt{gil.rei@tum.de}\\
	\end{minipage}
	%
	\vspace{1cm}\\
	%
	\rule{\textwidth}{1pt}
	%
	\begin{tabular}[t]{l}
		Institute for Computational Mechanics
		\\
		Prof.\@ Dr.--Ing.\@ W.\ A.\ Wall
		\\
		Technische Universit\"at M\"unchen
	\end{tabular}
	\hfill
	\begin{tabular}[t]{r}
		%   \copyright Lehrstuhl f\"{u}r Numerische Mechanik
		\DateOfSubmission
		\\
		Boltzmannstra\ss e 15
		\\
		85748 Garching b. M\"unchen (Germany)\\
	\end{tabular}
	\rule{\textwidth}{1pt}
	
\end{titlepage}

% ***************** Deckblatt Fakultaet MW *****************

%//##\cleardoublepage
\newpage
\thispagestyle{empty}
\null
\newpage
\thispagestyle{empty}
\begin{changemargin}{-1.0cm}{-1.0cm}
	
	\begin{flushright}
		\includegraphics[width=20mm]{fig/tum.png}
	\end{flushright}
	\vspace*{5mm}
	
	{\sffamily
		\Huge {\noindent
			Algebraic Multigrid Preconditioning for Contact and Mesh Tying Problems in Saddle Point Formulation}
		
		\vspace*{4cm}
		
		\large {\noindent
			Wissenschaftliche Arbeit zur Erlangung des Grades \\[1mm]
			B.Sc. \\[1mm]
			an der TUM School of Engineering and Design.
		}
		\vspace*{1.5cm}
		
		{%
			\normalsize
			\begin{onehalfspacing}
				\begin{raggedright}
					\begin{tabular}{ll}
						\textbf{Themenstellender} & Univ.-Prof.\@ Dr.-Ing.\@ Wolfgang\ A.\ Wall \\
						& Lehrstuhl f\"ur Numerische Mechanik \\[5mm]
						\textbf{Betreuer} & Dr.-Ing. Matthias Mayr  \\
						& (M.Sc. Gil Robalo Rei) \\[5mm]
						\textbf{Eingereicht von} & Philip Oesterle-Pekrun \\
						& +49 1522 6928328 \\[5mm]
						\textbf{Eingereicht am} & \DateOfSubmission~in Garching bei M\"unchen \\
					\end{tabular}
				\end{raggedright}
			\end{onehalfspacing}
		}
	}
\end{changemargin}
% *********************** Thesis Declaration **********************

\begin{changemargin}{-1.0cm}{-1.0cm}
	%//##\cleardoublepage
	%\insertblankpage
	\newpage
	\raggedright
	\begin{flushright}
		\includegraphics[width=20mm]{fig/tum.png}
	\end{flushright}
	
	\vspace*{1.5cm}
	
	{\LARGE \textsf{\textbf{\noindent 
				Erkl\"arung }}}
	\thispagestyle{empty}
	\vspace*{1.5cm}
	
	\begin{onehalfspacing}
		\begin{raggedright}
			\textsf{\noindent
				Ich versichere hiermit, dass ich die von mir eingereichte Abschlussarbeit selbstst\"andig verfasst
				und keine anderen als die angegebenen Quellen und Hilfsmittel benutzt habe.}
			
			\vspace{1.5cm}
			\flushleft
			
			\rule[-0,5cm]{\textwidth}{0,5pt}
			\flushleft
			\textsf{(Ort, Datum, Unterschrift)}
		\end{raggedright}
	\end{onehalfspacing}
	
	% === Insert image absolutely at position (x,y) in mm ===
	\begin{textblock}{50}(1.18,6.45) % width, (x,y) position in mm
		\includegraphics[width=70mm]{fig/ortDatum.png} % adjust path and width
	\end{textblock}
	
	\begin{textblock}{50}(7.5,6.22) % width, (x,y) position in mm
		\includegraphics[width=60mm]{fig/SignatureTranspBackground.png} % adjust path and width
	\end{textblock}
	
\end{changemargin}

% % Old version of English declaration

%   Hereby I confirm that this is my own work and I referenced all sources and
%   auxiliary means used and acknowledged any help that I have received from
%   others as appropriate.


%%%%%%%%%%%%%%%%%%%%%%%%%%%%%%%%%%%%%%%%%%%%%%%%%%%%%%%%%%%%%%%%%%%%%%%%%%%%%%%%
% Acknowledgement
%%%%%%%%%%%%%%%%%%%%%%%%%%%%%%%%%%%%%%%%%%%%%%%%%%%%%%%%%%%%%%%%%%%%%%%%%%%%%%%%
%//##\cleardoublepage

\pagestyle{plain}
\clearpage
\pagenumbering{roman}
\setcounter{page}{1}


\section*{Misc}
\thispagestyle{plain}
}

\comment{

	%%%%%%%%%%%%%%%%%%%%%%%%%%%%%%%%%%%%%%%%%%%%%%%%%%%%%%%%%%%%%%%%%%%%%%%%%%%%%%%%
	% Table of content
	%%%%%%%%%%%%%%%%%%%%%%%%%%%%%%%%%%%%%%%%%%%%%%%%%%%%%%%%%%%%%%%%%%%%%%%%%%%%%%%%
	%//##\cleardoublepage
	\clearpage
	%\setlength{\cftbeforesecskip}{2.5pt}
	\vspace*{-1.5cm}
	\tableofcontents
	
	%//# I think I will only list the nomenclature and not the ``symbols'' (same as nomenclature, no?), figures or tables
	% add lists as required - ask your supervisor
	
	%%%%%%%%%%%%%%%%%%%%%%%%%%%%%%%%%%%%%%%%%%%%%%%%%%%%%%%%%%%%%%%%%%%%%%%%%%%%%%%%
	% List of figures
	%%%%%%%%%%%%%%%%%%%%%%%%%%%%%%%%%%%%%%%%%%%%%%%%%%%%%%%%%%%%%%%%%%%%%%%%%%%%%%%%
	%//##\cleardoublepage
	\clearpage
	\listoffigures
	
	%%%%%%%%%%%%%%%%%%%%%%%%%%%%%%%%%%%%%%%%%%%%%%%%%%%%%%%%%%%%%%%%%%%%%%%%%%%%%%%%
	% List of tables
	%%%%%%%%%%%%%%%%%%%%%%%%%%%%%%%%%%%%%%%%%%%%%%%%%%%%%%%%%%%%%%%%%%%%%%%%%%%%%%%%
	\clearpage
	\listoftables
	

	
	%%%%%%%%%%%%%%%%%%%%%%%%%%%%%%%%%%%%%%%%%%%%%%%%%%%%%%%%%%%%%%%%%%%%%%%%%%%%%%%%
	% List of symbols
	%%%%%%%%%%%%%%%%%%%%%%%%%%%%%%%%%%%%%%%%%%%%%%%%%%%%%%%%%%%%%%%%%%%%%%%%%%%%%%%%
	%//##\cleardoublepage
	\insertblankpage
	\section*{List of Symbols}
	% either by hand or find fancy macro and let me know :o
	or
	% \section*{Nomenclature}
	
	% falls Arbeit in DEUTSCH verfasst werden soll
	% \section*{Nomenklatur} or
	% \section*{Bezeichnungen, Abk\"urzungen, Vereinbarungen}
	
}















\comment{
	
	%%%%%%%%%%%%%%%%%%%%%%%%%%%%%%%%%%%%%%%%%%%%%%%%%%%%%%%%%%%%%%%%%%%%%%%%%%%%%%%%
	%%% NOMENCLATURE %%%%%%%%%%%%%%%%%%%%%%%%%%%%%%%%%%%%%%%%%%%%%%%%%%%%%%%%%%%%%%%
	%%%%%%%%%%%%%%%%%%%%%%%%%%%%%%%%%%%%%%%%%%%%%%%%%%%%%%%%%%%%%%%%%%%%%%%%%%%%%%%%
	%//##\cleardoublepage
	\clearpage
	\section*{Nomenclature}
	% Table Layout
	\noindent
}
\comment{
	Notation:
	Upright non-bold symbols are operators or multiple-character things (e.g. Reynolds number Re)
	Italicized bold symbols are tensors (including first order tensors)
	Upright bold symbols are non-tensor matrices and vectors (e.g. stiffness matrix or nodal displacements vector)
}
\comment{
	\begin{tabbing}
		\textbf{Superscripts and Subscripts} \\
		(.)$^{(\mathcal{M})}$ \= \hspace{10em} \= Subdomain \\
		(.)$_0$ \= \hspace{10em} \= Subdomain
	\end{tabbing}
	\begin{tabbing}
		\textbf{Miscellaneous} \\
		%$\mathbb I _{\mathrm n,\mathrm d}$ \= \hspace{10em} \= n-th order identity tensor acting on $\mathbb R ^{\mathrm d}$ \\ %//## maybe a bit over the top but lets see
		$\mathbb I _{\mathrm n}$ \= \hspace{10em} \= 2nd order identity tensor acting on $\mathbb R ^{\mathrm n}$ \\
		$\delta_{ij}$ \= \hspace{10em} \= Kronecker delta \\
		$\left\| \cdot \right\|_p$ \= \hspace{10em} \= $\mathrm L^p$ norm \\
		$\left\| \cdot \right\|$ \= \hspace{10em} \= Any consistent norm \\
	\end{tabbing}
	\begin{tabbing}
		\textbf{Solid Mechanics and FEM} \\
		$\bm{X}$ \= \hspace{10em} \= Material points in the reference configuration \\
		$\bm{x}$ \= \hspace{10em} \= Material points in the current configuration \\
		$\bm{u}$ \= \hspace{10em} \= Solid deformation \\
	\end{tabbing}
}
%%%%%%%%%%%%%%%%%%%%%%%%%%%%%%%%%%%%%%%%%%%%%%%%%%%%%%%%%%%%%%%%%%%%%%%%%%%%%%%%
%%% CONTENT %%%%%%%%%%%%%%%%%%%%%%%%%%%%%%%%%%%%%%%%%%%%%%%%%%%%%%%%%%%%%%%%%%%%
%%%%%%%%%%%%%%%%%%%%%%%%%%%%%%%%%%%%%%%%%%%%%%%%%%%%%%%%%%%%%%%%%%%%%%%%%%%%%%%%
%//##\cleardoublepage
\clearpage

\pagenumbering{arabic}

\pagestyle{fancyplain}
\fancyhf{}
\fancyfoot[C]{\fancyplain{}{\bfseries\thepage}}
\fancyhead[RO]{\fancyplain{}{\Author}}
\fancyhead[LO]{\fancyplain{}{\thesistype:\ \shorttitle}}

\setboolean{@twoside}{false} %//## HERE TO CHANGE BETWEEN ASYMMETRIC VS SYMMETRIC MARGINS
\fancyhead[RO,LE,LO]{\fancyplain{}{}}
\fancyhead[CE]{\fancyplain{}{\leftmark}}
\fancyhead[CO]{\fancyplain{}{\rightmark}}





\pagestyle{fancyplain}

%%%%%%%%%%%%%%%%%%%%%%%%%%%%%%%%%%%%%%%%%%%%%%%%%%%%%%%%%%%%%%%%%%%%%%%%%%%%%%%%
% Section 1
%%%%%%%%%%%%%%%%%%%%%%%%%%%%%%%%%%%%%%%%%%%%%%%%%%%%%%%%%%%%%%%%%%%%%%%%%%%%%%%%
\section{FEM Solver Implementation} \label{sec:intro}
Full weak form:
\begin{equation}
	-\delta W = \underbrace{\int_{\Omega_0} \bm S : \delta \bm E \mathrm d V}_{-\delta W_{\text{int}}} \underbrace{- \int_{\Omega_0} \hat{\bm b}_0^T \delta \bm u \mathrm d V - \int_{\Gamma_{0; \bm \sigma}} \hat{\bm t}_0^T \delta \bm u \mathrm d A}_{-\delta W_{\text{ext}}} \overset{!}{=} 0 .
\end{equation}
%
We neglect $\delta W_{\text{ext}}$ for now, and focus on internal work:
\begin{equation} \label{eq:dwInt}
	-\delta W_{\text{int}} = \int_{\Omega_0} \bm S : \delta \bm E \mathrm d V \equiv \int_{\Omega_0} \delta \bm E : \bm S \mathrm d V \overset{!}{=} 0 ,
\end{equation}
due to symmetry of both tensors.

We use the linear (engineering) strain
\begin{equation} \label{eq:kinematicEngStrain}
	\bm E = \text{symmetrize}(\nabla \bm u) = \frac{1}{2} (\nabla \bm u + (\nabla \bm u)^T) ,
\end{equation}
as well as the linear St.Venant-Kirchoff constitutive relation
\begin{equation}
	\bm S = \bm C_{\text{VK}} : \bm E ,
\end{equation}
\begin{equation}
	(C_{\text{VK}})_{ijkl} = \lambda \delta_{ij} \delta{kl} + \mu (\delta_{ik} \delta_{jl} + \delta_{il} \delta_{jk}) ,
\end{equation}
with the Lam\'e constants $\lambda$ and $\mu$.

\subsubsection{Discretization within an element}
We use the discretized quantities (technically they should have an "h" subscript, but we leave it away here because we do not handle the analytical quantities anyways)
\begin{align}
	u_i &= \sum_{k = 1}^{\text{nnode}} \mathrm N_k \mathrm d_{\text{nodeldof2dof}(k, i)} , \\
	\delta u_i &= \sum_{k = 1}^{\text{nnode}} \mathrm N_k \delta \mathrm d_{\text{nodeldof2dof}(k, i)} , \\
	x_i &= \sum_{k = 1}^{\text{nnode}} \mathrm N_k \mathrm X_{\text{nodeldof2dof}(k, i)} ,
\end{align}
where
\begin{equation}
	\text{nodeldof2dof}(k, i) = k \cdot \text{nnode} + i
\end{equation}
is the mapping from node $k$ and node-local (node-scoped) dof $i$ to the corresponding element-global (element-scoped) dof. The inverse of this mapping is
\begin{equation}
	\text{dof2nodeldof}(g) = \{\text{floor}(g/\text{ndofn}), \text{mod}(g, \text{ndofn})\} .
\end{equation}
%
We will henceforth use the shorthand notation
\begin{equation}
	\mathrm q_{(k, j)} := \mathrm q_{\text{nodeldof2dof}(k, i)} ,
\end{equation}
for any non-tensor vector quantity $\mathrm q$ as above.

Now, given the need for the gradient of the unknown displacement in (\ref{eq:kinematicEngStrain}), we need to discretize it. In index notation,
\begin{equation}
	\frac{\partial u_i}{\partial x_j} = \frac{\partial (\mathrm N \mathrm d)_i}{\partial x_j} = \sum_{k = 1}^{\text{nnode}} \frac{\partial \mathrm N_k}{\partial x_j} \mathrm d_{\text{nodeldof2dof}(k, i)} .
\end{equation}
%
Now, we can find the gradient of the shape functions from the parametric coordinates and (inverse) Jacobian:
\begin{equation}
	\frac{\partial \mathrm N_k}{\partial x_j} = \sum_{l = 1}^{\text{ndofn}} \frac{\partial \mathrm N_k}{\partial \xi_l} \frac{\partial \xi_l}{\partial x_j}.
\end{equation}
To find the inverse Jacobian, we need the Jacobian:
\begin{equation}
	J_{lj} = \frac{\partial x_j}{\partial \xi_l} = \frac{\sum_{k = 1}^{\text{nnode}} \mathrm N_k \mathrm X_{\text{nodeldof2dof}(k, j)}}{\partial \xi_l} = \sum_{k = 1}^{\text{nnode}} \frac{\mathrm N_k}{\partial \xi_l} \mathrm X_{\text{nodeldof2dof}(k, j)} .
\end{equation}
%
Ok, so, that was easy enough with the Jacobian. It is an isolated task. But, we now look at the rest of the stuff, i.e. (\ref{eq:dwInt}). We start from (\ref{eq:kinematicEngStrain}) basically.
\begin{equation}
	E_{ij} = frac{1}{2} (\frac{\partial u_i}{\partial x_j} + \frac{\partial u_j}{\partial x_i}) =
	frac{1}{2} (\frac{\partial N_i}{\partial x_j} + \frac{\partial u_j}{\partial x_i})	
\end{equation}







%%%%%%%%%%%%%%%%%%%%%%%%%%%%%%%%%%%%%%%%%%%%%%%%%%%%%%%%%%%%%%%%%%%%%%%%%%%%%%%%
%%% Bibliography %%%%%%%%%%%%%%%%%%%%%%%
%%%%%%%%%%%%%%%%%%%%%%%%%%%%%%%%%%%%%%%%%%%%%%%%%%%%%%%%%%%%%%%%%%%%%%%%%%%%%%%%
\comment{
	\cleardoublepage
	%\bibliographystyle{plain}
	%\bibliography{00_Bibliography.bib}
	\addcontentsline{toc}{section}{References}
	\printbibliography
	
	%%%%%%%%%%%%%%%%%%%%%%%%%%%%%%%%%%%%%%%%%%%%%%%%%%%%%%%%%%%%%%%%%%%%%%%%%%%%%%%%
	%%%%%%%%%%%%%%%%%%%%%%%%%%%%%%%%%%%%%%%%%%%%%%%%%%%%%%%%%%%%%%%%%%%%%%%%%%%%%%%%
	%%% END DOCUMENT
	%%%%%%%%%%%%%%%%%%%%%%%%%%%%%%%%%%%%%%%%%%%%%%%%%%%%%%%%%%%%%%%%%%%%%%%%%%%%%%%%
	%%%%%%%%%%%%%%%%%%%%%%%%%%%%%%%%%%%%%%%%%%%%%%%%%%%%%%%%%%%%%%%%%%%%%%%%%%%%%%%%
}
\end{document}
